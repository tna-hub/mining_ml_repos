\documentclass[../main.tex]{subfiles}

\begin{document}
    The increasing popularity of Machine Learning (ML) is generating growing interest for developers.
    The multitude of programing languages, libraries and available resources allow them to break into machine learning and build their own ML algorithm from scratch.
    However, ML models are tightly connected to their training dataset, involving a slightly different development process from that of traditional software.
    To understand the impact and challenges of integrating external libraries and dataset to Machine learning applications, we studied 42505 GitHub ML projects from Papers with Code having Python as main programing language.
    We present a detailed analysis of how these projects integrate external libraries and data to the repositories though GitHub.
    We found that there is a strong association between the libraries, and that although there are different methods used to integrate datasets to repositories, these methods tend to have an impact on the evolution of the code, creating a strong correlation between datasets and the core ML code that can be hard to track in common versionning systems such as Github.
\end{document}